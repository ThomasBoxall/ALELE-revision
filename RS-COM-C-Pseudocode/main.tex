\documentclass[a4paper,11pt]{article}
\usepackage[margin=0.8in]{geometry}
\usepackage{xcolor}
\usepackage{graphicx} %package to manage images
\graphicspath{ {./images/} }
\usepackage{fancyvrb}
\usepackage{fvextra}

\title{Pseudocode}
\author{Revision sheet}
\date{}

\usepackage{fancyhdr}
\pagestyle{fancy}
\fancyhead{} % clear all header fields
\renewcommand{\headrulewidth}{0pt} % no line in header area
\fancyfoot{} % clear all footer fields
\renewcommand{\footrulewidth}{0.4pt}
\fancyfoot[C]{\thepage} % page number in centre of the page
\fancyfoot[R]{\footnotesize Thomas Boxall \\ Pseudocode taken from OCR guide and previous papers} % right hand footer has author name on top line and images reference on bottom line
\fancyfoot[L]{\footnotesize Pseudocode \\ Revision sheet} % left hand footer has title of document on top line and 'Revision Sheet' on bottom line


\begin{document}

\maketitle
\thispagestyle{fancy}

% CONTENTS OF THE REVISION SHEET HERE
\section{Syntax}
Pseudocode doesn't use curly brackets, it uses indentation and end statements for multi-line code sections (for example \verb|endif|).
\subsection{Comments}
Lines which are to be commented out begin with a double slash, \verb|//|.
\subsection{Variables}
Variables are declared the first time they are assigned. Data types do not need to be declared. For example:\\
\verb|x = 3| would assign 3 to variable called x.\\
\verb|name = 'bob'| would assign `bob' to variable called name.
\subsubsection{Casting}
Variables can be typecast using the functions shown below.\\
\verb|str(3)| returns \verb|'3'|.\\
\verb|int('3')| returns \verb|3|.\\
\verb|float('3.14')| returns \verb|3.14|.
\subsection{Arrays}
Arrays are zero based and are declared and assigned in the following way:\\
\verb|array names[5]|\\
\verb|names[0]='Ben'|
\subsubsection{Two Dimensional Arrays}
These are declared and assigned as followed.\\
\verb|array board[8,8]|\\
\verb|board[0,0]='rook'|
\subsection{Input And Output}
\subsubsection{Input}
This is done as follows\\
\verb|variableName = input('string to appear on screen')|
\subsubsection{Output}
This is done as follows\\
\verb|print(string)|\\
For example\\
\verb|print('hello world')|
\subsection{Iteration}
\subsubsection{Count Controlled}
The example shows a for loop which will run 8 times (0 to 7 inclusive).
\begin{Verbatim}[breaklines=true, breakanywhere=true]
for i=0 to 7
    print('Hello world')
next i
\end{Verbatim}
\subsubsection{Condition Controlled}
The examples below both use the same situation, input validation. The first example uses a while loop.
\begin{Verbatim}[breaklines=true, breakanywhere=true]
while answer != 'computer'
    answer = input ('What is the password?')
endwhile
\end{Verbatim}
The second example uses a do while loop.
\begin{Verbatim}[breaklines=true, breakanywhere=true]
do
    answer = input('What is the password?')
until answer == 'computer'
\end{Verbatim}
\subsection{Selection}
In the examples in the following sections, the code is looking at the contents of the variable \verb|entry| and outputting a string based off of its contents.
\subsubsection{If}
The example below uses \verb|if|, \verb|elseif| and \verb|else|.
\begin{Verbatim}[breaklines=true, breakanywhere=true]
if entry == 'a' then
    print('You selected A')
elseif entry == 'b' then
    print('You selected B')
else
    print('Unrecognised Selection')
endif
\end{Verbatim}
\subsubsection{Switch}
The examples below use a \verb|switch| statement.
\begin{Verbatim}[breaklines=true, breakanywhere=true]
switch entry:
    case 'A':
        print('You selected A')
    case 'B':
        print('You selected B')
    default:
        print('Unrecognised selection')
endswitch
\end{Verbatim}
\subsection{Operators}
These are the same as they are in other programming languages.
\subsection{String Handling}
To get the length of a string, use the following syntax:\\
\verb|stringname.length|\\
To get a substring, use the following syntax (nb. strings are zero-based - like arrays):\\
\verb|stringname.subString(startingPosition, numberOfCharacters)|
\subsection{Subroutines}
\subsubsection{Functions}
\begin{Verbatim}[breaklines=true, breakanywhere=true]
function triple (number)
    return number * 3
endfunction
\end{Verbatim}
\subsubsection{Procedure}
\begin{Verbatim}[breaklines=true, breakanywhere=true]
procedure greeting(name)
    print('Hello' + name)
endprocedure
\end{Verbatim}
\subsubsection{Calling From The Main Program}
\begin{Verbatim}[breaklines=true, breakanywhere=true]
greeting('Hamish')
\end{Verbatim}
The code above would print \verb|Hello Hamish| to the screen. 
\subsubsection{Parameters}
By default, these are passed by value. If the question doesn't specify, assume it means pass by value.\\
\verb|parName:byVal| - by value declaration.\\
\verb|parName:byRef| - by reference declaration.
\subsection{Files}
The following program will print out the contents of \verb|sample.txt|. The same code can be adapted to save the contents of the file to an array or string for example. 
\begin{Verbatim}[breaklines=true, breakanywhere=true]
myFile = openRead('sample.txt')
while NOT myFile.endOfFile()
    print(myFile.readLine())
endwhile
myFile.close()
\end{Verbatim}
To write to a file, use the following syntax
\begin{Verbatim}[breaklines=true, breakanywhere=true]
myFile = openWrite('sample.txt')
myFile.writeLine('Hello World')
myFile.close()
\end{Verbatim}
\subsection{Object Oriented Pseudocode}
\subsubsection{Class Declaration and Inheritance}
The code below declares a class called \verb|Pet| with an attribute \verb|name| and a method called \verb|new| - this is the classes constructor.
\begin{Verbatim}[breaklines=true, breakanywhere=true]
class Pet
    private name
    public procedure new(givenName)
        name = givenName
    endprocedure
endclass
\end{Verbatim}
Inheritance is denoted by the \verb|inherits| keyword. Methods from the super class will be called with the keyword \verb|super|. For example, \verb|super.methodName(parameters)| The example below shows a subclass of \verb|Pet| called \verb|Dog| (notice the constructor for the superclass uses the \verb|super.methodName| syntax).\\
\begin{Verbatim}[breaklines=true, breakanywhere=true]
class Dog inherits Pet
    private breed
    public procedure new(giveName, givenBreed)
        super.new(givenName)
        breed = givenBreed
    endprocedure
endclass
\end{Verbatim}
\subsubsection{Methods and Attributes}
Both methods and attributes can be assumed to be public unless otherwise stated. Where the access level is relevant to the question, it will always be explicit in the code (denoted by the keywords \verb|public| and \verb|private|). The example below shows a public procedure.
\begin{Verbatim}[breaklines=true, breakanywhere=true]
public procedure setAttempts(number)
    attempts=number
endprocedure
\end{Verbatim}
Methods will be called using the following syntax:
\begin{Verbatim}[breaklines=true, breakanywhere=true]
player.setAttempts(5)
\end{Verbatim}


\end{document}


\begin{Verbatim}[breaklines=true, breakanywhere=true]

\end{Verbatim}

comments
variables
    casting
input output
    input
    output
iteration
    count controlled
    condition controlled
selection
    if else
    switch case
logical operators
string handling
subroutines
    function
    calling it
arrays
Files
    reading to
    writing from
    other bits
oop
    methods and attributes
    constructors and inheritance
    
then get some examples of longer exam questions from previous papers.