\chapter{Evaluation}
Once I had fully constructed my circuit, I am able to connect it to the power supply and evaluate what I have produced against my success criteria.
\section{Evaluating against Specification}
\begin{itemize}
    \item \textbf{The voltmeter will measure in volts and range from 0V to 5V, in 0.1V increments.}\\ When the voltmeter works, this is achieved. However as the voltmeter doesn't work reliably this point \tempText{red}{cannot be marked as achieved}.
    \item \textbf{The voltmeter will have a sample rate of 1KHz $\pm 400Hz$ times per second. This will be produced by Schmitt Astable with a frequency of 1KHz.}\\ From testing the clock subsystem, I know that the frequency of the clock is $1.36KHz$. This is a suitable deviation from the original success criteria (as it is an increase of less than 400Hz), meaning the clock pulses 1360 times per second. Therefore this point \tempText{Green}{can be marked as achieved}.
    \item \textbf{The voltmeter is accurate to within $\pm0.1V$.}\\ From testing the final circuit, I have found out that when it works, it is accurate within my specified bounds but when it doesn't work it is extremely inaccurate. Therefore this point \tempText{orange}{can be marked as partially achieved}.
    \item \textbf{The voltmeter takes less than 0.5 seconds to adjust its output once a change in voltage is detected.}\\ Despite the fact that the voltmeter was unreliable, I was able to measure the frequency of the clock. This frequency would mean that the clock could complete a full ramp in less than 0.5 seconds. Therefore this point \tempText{Green}{marked as achieved}.
    \item \textbf{The voltmeter should take a +5v, 0V and -5V (within ±0.5V) input.}\\ These are the inputs required for the DAC subsystem, therefore this point \tempText{Green}{can be marked as achieved}.
    \item \textbf{The voltmeter will be easy to use.}\\ From testing and asking the opinions of those who have tested my voltmeter for me, I have found that the voltmeter is easy to use. Therefore this point \tempText{Green}{can be marked as achieved}.
    \item \textbf{The voltage will output to two 7-segment displays; one for the units and one for the tenths.} \\ As seen in my full layout image, there are two 7-segment displays, one of which is used for units and the other for tenths. Therefore this point \tempText{Green}{can be marked as achieved}.
    \item \textbf{The entire circuit will only require human input to connect the analogue voltage.}\\ When the circuit is working, this point is achieved. However as the circuit doesn't work all of the time, this point \tempText{orange}{can be marked as partially complete}.
    \item \textbf{The voltmeter should be as efficient as possible, reducing heat dissipated to the environment.}\\ As my circuit is using a low voltage and current input, there isn't much energy to be dissipated. This low energy input requirement is partially due to the fact that I chose to use CMOS chips rather than TTL. TTL require a much higher operating voltage and current therefore they dissipate a much higher amount of energy. Therefore this point \tempText{Green}{can be marked as achieved}.
\end{itemize}
From evaluating against my specification, I can conclude that this project was not a complete success as only 66\% of the specification points were met. \newline For my project to have been a success, I would have needed an 75\% achievement rate of my success criteria. This would have meant 7 of the points would have to be achieved.

\section{Improvements}
To improve this project, I would troubleshoot further into the cause of the latches not clocking at the right time, then work to fix that issue. Assuming that this is the only issue I come into, then that would allow me to mark the remaining 4 success criteria points as achieved.\newline
A further improvement I could make is to take my layout on breadboards, and use a digital tool to convert it into a PCB. This would reduce the overall physical layout of the circuit as well as reduce the chance for a component to be knocked out. Furthermore, this would improve the noise performance of the circuit and allow for a better grounding arrangement. This would also allow me to implement a better timing system as it would be simpler to construct.

\section{Conclusion}
In conclusion, my system did not meet all of the specification points I put together before constructing it. This means it is not a functioning voltmeter. This was an extremely challenging project however once I understood how the subsystems worked, I was able to design and build them using the knowledge I gained from the theory modules of the course. The most challenging aspect was troubleshooting the many issues I ran into, especially the noise on the output from the DAC.