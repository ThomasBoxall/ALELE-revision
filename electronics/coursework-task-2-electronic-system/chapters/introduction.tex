\chapter{Introduction}
\section{Introduction To The Project}
I chose this project with the idea of measuring the voltage of batteries up to 5V. The inspiration for this project came from seeing a battery tester and wondering how it works. This project also sounded like a challenge as it is combining lots of the different subsystems I learnt about in the first year of the A level course into one big system.
The system will begin when the power to the circuit is turned on; it will work continuously, displaying the voltage connected until the system is powered down. If no voltage is connected, it will display 0V.
\section{Definition Of A Voltmeter}
Before researching the different types of voltmeter and how they worked, I thought it would be useful to find out the definition of a voltmeter. Wikipedia\footnote{https://en.wikipedia.org/wiki/Voltmeter} defines a Voltmeter as:
\begin{table}[H]
    \begin{tabularx}{0.9\textwidth} { c | X }
         &  \textsc{A voltmeter is an instrument used for measuring electric potential difference between two points in an electric circuit. It is connected in parallel. It usually has a high resistance so that it takes negligible current from the circuit.}
    \end{tabularx}
\end{table}
