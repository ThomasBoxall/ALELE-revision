\documentclass[a4paper,11pt, twocolumn]{article}
\usepackage[margin=0.8in]{geometry}
\usepackage{xcolor}
\usepackage{graphicx} %package to manage images
\graphicspath{ {./images/} }
\usepackage{float}
\usepackage{tabularx}

\title{1.4.1 Data Types}
\author{Revision sheet}
\date{}

\usepackage{fancyhdr}
\pagestyle{fancy}
\fancyhead{} % clear all header fields
\renewcommand{\headrulewidth}{0pt} % no line in header area
\fancyfoot{} % clear all footer fields
\renewcommand{\footrulewidth}{0.4pt}
\fancyfoot[C]{\thepage} % page number in centre of the page
\fancyfoot[R]{\footnotesize Thomas Boxall \\ } % right hand footer has author name on top line and images reference on bottom line
\fancyfoot[L]{\footnotesize 1.4.1 Data Types \\ Revision sheet} % left hand footer has title of document on top line and 'Revision Sheet' on bottom line


\begin{document}

\maketitle
\thispagestyle{fancy}

% CONTENTS OF THE REVISION SHEET HERE
\section{Primitive Data Types}
A formal description of the type of data being stored or manipulated in a program. Important as they determine the operations which can be performed on the data. All programming languages support the same basic data types, they might be handled differently from language to language.
\subsection{Examples}
Below are a number of examples of data types and their uses.
\subsubsection{String}
A collection of characters including spaces and common keyboard symbols. Can also contain numbers but these will be handled as text characters not numbers.
\subsubsection{Integer}
A whole positive or negative number. Cannot have decimal places.
\subsubsection{Real}
Also known as floating point numbers. They contain decimal places and can either be positive or negative.
\subsubsection{Char}
This is a single character; which can be any number, letter or symbol. 
\subsubsection{Boolean}
This can only have one of two values, true or false. Different languages use different capitalisation of the values.
\subsection{More complex data types}
Most languages will also have a number of other data types. These could be composite types, which are made up of a number of primitive types put together (eg. an array of integers).
\subsection{Pointers}
These are built in data types which some languages have that are uses to point to a value or object located in computer memory.
\subsection{Null}
If a data type contains nothing, then it contains a null value.

\section{Number Systems}
There are a number of different number systems and different methods to convert between them. 
\subsection{Denary (Base 10)}
Used most commonly, this is the one most people learn.
\begin{table}[H]
    \centering
    \begin{tabularx}{0.9\linewidth}{c c c c}
        1000 & 100 & 10 & 1 \\
        4 & 2 & 5 & 1
    \end{tabularx}
\end{table}
\noindent The total of the numbers above would be calculated in the following way:\\
$4251=(1000\times 4) + (100 \times 2) + (10 \times 5) + (1 \times 1)$\\
Denary is also known as base 10, this means each column can have one of ten possible values (0, 1, 2, 3, 4, 5, 6, 7, 8, 9)
\subsection{Binary (Base 2)}
This is base 2, this means each column can have one of two possible values (0, 1). The columns are also different. Moving from right to left, the columns double each time.
\begin{table}[H]
    \centering
    \begin{tabularx}{0.9\linewidth}{c c c c c c c c}
        128 & 64 & 32 & 16 & 8 & 4 & 2 & 1 \\
        1 & 0 & 1 & 1 & 0 & 0 & 1 & 1
    \end{tabularx}
\end{table}
\noindent The largest value which can be stored in binary is $11111111_2$ or $255_{10}$. \\
Binary will be looked at in greater detail throughout this revision sheet.
\subsection{Hexadecimal (Base 16)}
Also known as Hex. Using this method, numbers up to 255 can be stored in two characters. This is used a lot in computing, especially in graphics and website development. There are 16 columns (getting bigger in value from right to left)\\
F E D C B A 9 8 7 6 5 4 3 2 1 0 
\subsection{Converting Between Number Systems}
\subsubsection{Binary To Denary}
Add together all the columns in which there is a 1. Using the example shown in the binary section, the total would be 179.
\subsubsection{Denary To Binary}
This is the reverse of binary to denary. Work from right to left seeing if the value will fit into the column, if it won't then mark down an zero and move onto the next.
\subsubsection{Denary to Hex}
The easiest way to do this is to go via Binary. Convert the number into binary, then split the binary into two nibbles. The values inputted in the previous step don't need to change. With the two nibbles of (4, 2, 1, 0), convert each of them back into denary, giving two individual digits, then convert each of those into Hex. 

\section{Negative Binary}
There are two ways to represent negative binary.
\subsection{Sign And Magnitude}
The MSB is used to store the sign (this means it can't be used as a number). This is the easy method as +75 can be written the same as -75. For example
\begin{table}[H]
    \centering
    \begin{tabularx}{0.9\linewidth}{c | c c c c c c c c}
         & - & 64 & 32 & 16 & 8 & 4 & 2 & 1 \\
        +75 & 0 & 1 & 0 & 0 & 1 & 0 & 1 & 1 \\
        -75 & 1 & 1 & 0 & 0 & 1 & 0 & 1 & 1
    \end{tabularx}
\end{table}
\noindent Sign and Magnitude has problems: the size of the number is what it was before as we only have one less bit; there are two different data types in one number (computer struggles to understand this); more difficult for computer to do calculations. 
\subsection{2's Complement}
In this method, the most significant bit becomes -128. Then we add each of the other bits to the MSB value to reach our desired value. For example -75 in 2's complement can be seen below
\begin{table}[H]
    \centering
    \begin{tabularx}{0.9\linewidth}{c | c c c c c c c c}
         & -128 & 64 & 32 & 16 & 8 & 4 & 2 & 1 \\
        -75 & 1 & 0 & 1 & 1 & 0 & 1 & 0 & 1
    \end{tabularx}
\end{table}

\section{Binary Maths}
To do maths with floating point binary, the binary point has to be in the same place for all numbers involved.
\subsection{Binary Addition}
This works in a very similar way to denary addition.
It is shown below
\begin{table}[H]
    \centering
    \begin{tabularx}{0.9\linewidth}{c | c c c c c c c c}
         & 128 & 64 & 32 & 16 & 8 & 4 & 2 & 1 \\
         n1 & 0 & 1 & 1 & 0 & 0 & 1 & 1 & 1 \\
        +n2 & 0 & 0 & 0 & 1 & 0 & 1 & 1 & 0 \\
        \hline
        result & 0 & 1 & 1 & 1 & 1 & 1 & 0 & 1 \\
        \hline
        carry &  &  &  &  & 1 & 1 &  & 
    \end{tabularx}
\end{table}
\noindent There are three basic cases. These cases can be expanded for as many numbers as needed.\\
1+0=1\\
1+1=0 carry 1\\
1+1+1 = 1 carry 1
\subsubsection{Overflow Errors}
These are caused when there is a 1 in the carry for the furthest left hand column. 
\subsection{Binary Subtraction}
There are basic rules to binary subtraction\\
0-0=0\\
1-0=1\\
1-1=0\\
0-1=1 (borrow 1)

\section{Decimal Binary Numbers}
There are two ways to represent decimal numbers in binary. 
\subsection{Fixed Point}
\begin{table}[H]
    \centering
    \begin{tabularx}{0.9\linewidth}{c c c c c c c c c}
        8 & 4 & 2 & 1 & . & $\frac{1}{2}$ & $\frac{1}{4}$ & $\frac{1}{8}$  & $\frac{1}{16}$\\
        0 & 1 & 0 & 1 & . & 1 & 1 & 0 & 0
    \end{tabularx}
\end{table}
The example above shows an 8-bit representation of fixed point binary with the value 5.75. Exactly the same maths and logical operations can be performed on this as can be done with integer binary - it is important to make sure that the binary point is in the same place in both numbers though. To convert denary into fixed point binary is done exactly the same as with integer binary. There can be as many or as few bits before and after the binary point.
\subsection{Floating Point}
This is similar to fixed point, in that you can represent decimal numbers in binary however floating point has greater range and precision as the binary point can move. There are two parts to a floating point binary number - the mantissa and the exponent. The mantissa is the number itself and the exponent is the value which the binary point has to be moved by to get to the original number. The binary point, unless specified, will always be between the left most and second left most bit of the mantissa.\\
For example, the floating point binary number \verb|0.101101 0011| has a 7 bit mantissa and 4 bit exponent. The exponent has a value of 3 in denary therefore we need to move the binary point three places to the right, giving us \verb|0101.101|. We now treat the mantissa as we would any other binary number to convert to denary, giving us \verb|5.625|. This process is the same for a negative exponent, except for the fact that the binary point moves left. 
\subsubsection{Normalisation}
Normalisation is the process of moving the binary point so the most precise number possible can be achieved. This is achieved using the same method outlined above, but in reverse. A positive normalised mantissa will always start with \verb|0.1| and a negative normalised mantissa will always start with \verb|1.0|. 
\subsubsection{Maths}
Floating point addition and subtraction work exactly the same as standard binary addition and subtraction, however the binary point needs to line up in the two numbers.

\section{Bitwise Manipulation And Masks}
\subsection{Shifts}
Binary numbers can be multiplied together or divided using a combination of shifts and addition. 
\subsubsection{Logical Shift}
All the bits move left or right. The example below shows a logical right shift which forces the least significant bit into a `carry bit' and the most significant bit padded out with a 0. Logical right shifts are useful for examining the contents of the least significant bit where it can be tested then the program branched.
\begin{table}[H]
    \centering
    \begin{tabularx}{0.9\linewidth}{c | c c c c c c c c | | X}
         &  &  &  &  &  &  &  &  & carry bit\\
         \hline
        before & 1 & 0 & 1 & 1 & 0 & 0 & 0 & 1 & - \\
        after & 0 & 1 & 0 & 1 & 1 & 0 & 0 & 0 & 1 
    \end{tabularx}
\end{table}
\subsubsection{Arithmetic Shift}
This is fundamentally the same as a logical shift except the sign bit stays the same.
\begin{table}[H]
    \centering
    \begin{tabularx}{0.9\linewidth}{c | c c c c c c c c | | X}
         &  &  &  &  &  &  &  &  & carry bit\\
         \hline
        before & 1 & 0 & 1 & 1 & 0 & 0 & 0 & 1 & - \\
        after & 1 & 1 & 0 & 1 & 1 & 0 & 0 & 0 & 1 
    \end{tabularx}
\end{table}
\subsubsection{Circular Shifts}
This is useful for performing shifts in multiple bytes. In a right circular shift, the value in the least significant bit is moved into the carry bit and the carry bit is moved into the most significant bit. 
\begin{table}[H]
    \centering
    \begin{tabularx}{0.9\linewidth}{c | c c c c c c c c | | X}
         &  &  &  &  &  &  &  &  & carry bit\\
         \hline
        before & 1 & 0 & 0 & 0 & 1 & 1 & 0 & 1 & 0 \\
        after & 0 & 1 & 0 & 0 & 0 & 1 & 1 & 0 & 1
    \end{tabularx}
\end{table}
\subsection{Masks}
The output from a mask would be the same as if the boolean number was applied to that logic gate in individual bits. 
\begin{table}[H]
\begin{tabularx}{0.45\textwidth}{XXllll}
 &  & NOT & AND & OR & XOR \\
Input & A & 1010 & 1010 & 1010 & 1010 \\
Input & B &  & 1100 & 1100 & 1100 \\ \cline{3-6} 
Result &  & 0101 & 1000 & 1110 & 0110 \\ \cline{3-6} 
\end{tabularx}
\end{table}

\section{Character Sets}
A character set is \textit{a complete set of the characters and their number codes that can be recognised by a computer system}.
There are two character sets which are commonly used: ASCII and Unicode.
\subsection{The Need For Character Sets}
Character sets are needed so that many different computers or connected devices are able to interpret messages and data transmitted between them, and translate them back to the original message.
\subsection{ASCII}
The \textit{American Standard Code for Information Interchange} was one of the first mainstream character sets adopted into computing. It uses 7 bits, representing 128 different values. The first 32 of these values are non-printing control characters (eg, \verb|space| or \verb|NULL|). The remainder of the characters are used for printing characters (eg, \verb|A| or \verb|-|), apart from the last one which is the \verb|DEL| character.
\subsubsection{8-Bit ASCII}
After some time, a larger character set was needed, the solution to this was to add a leading zero to all of the 7-bit ASCII character values. This turned ASCII into an 8-bit character set and allowed another 128 characters to be included in the set. This still wasn't big enough - Unicode was developed to solve this problem.
\subsection{Unicode}
By the 1980s, around the world there were several character sets which were being developed independently and were not compatible with each other. A new character set was developed by the Unicode Consortium, called Unicode (UTF-16) which is 16 bits and can represent 65536 different characters. The first 128 characters were kept the same for backwards compatibility with ASCII. A further 32-bit (UTF-32) Unicode character set was also developed, this had more characters in it. Whilst the addition of a globally recognised character set was good; the new character set had an increased size therefore files using it would have an increased sizes therefore transmission of these files would take longer. 



\end{document}


t(a) Primitive data types, integer, real/floating point, character, string and Boolean.
t(b) Represent positive integers in binary.
t(c) Use of sign and magnitude and two’s complement to represent negative numbers in binary.
t(d) Addition and subtraction of binary integers.
t(e) Represent positive integers in hexadecimal.
t(f) Convert positive integers between binary hexadecimal and denary.
t(g) Representation and normalisation of floating point numbers in binary.
t(h) Floating point arithmetic, positive and negative numbers, addition and subtraction.
t(i) Bitwise manipulation and masks: shifts, combining with AND, OR, and XOR.
(j) How character sets (ASCII and UNICODE) are used to represent text.
