\documentclass[a4paper,11pt, twocolumn]{article}
\usepackage[margin=0.8in]{geometry}
\usepackage{xcolor}
\usepackage{graphicx} %package to manage images
\graphicspath{ {./images/} }

\title{1.1.3 Input, Output And Storage}
\author{Revision sheet}
\date{}

\usepackage{fancyhdr}
\pagestyle{fancy}
\fancyhead{} % clear all header fields
\renewcommand{\headrulewidth}{0pt} % no line in header area
\fancyfoot{} % clear all footer fields
\renewcommand{\footrulewidth}{0.4pt}
\fancyfoot[C]{\thepage} % page number in centre of the page
\fancyfoot[R]{\footnotesize Thomas Boxall \\ } % right hand footer has author name on top line and images reference on bottom line
\fancyfoot[L]{\footnotesize 1.1.3 Input, Output And Storage \\ Revision sheet} % left hand footer has title of document on top line and 'Revision Sheet' on bottom line


\begin{document}

\maketitle
\thispagestyle{fancy}

% CONTENTS OF THE REVISION SHEET HERE
\section{Input Devices}
There are many different input devices: keyboard; mouse; trackpad; graphics tablet; microphone; scanner; NFC reader; touchscreen.
\subsection{Specialised Input Devices}
Some people are not able to use standard input devices due to a physical disability. There are three additional input methods which they can use: sip/puff switch; foot switches; braille keyboard.
\subsection{Barcodes}
Barcodes represent data in a machine readable form. Traditionally they use black lines varying in width and spacing. This can be interpreted by a computer as a string of numbers. Many systems make use of a \textit{check digit} which is an additional digit which can be calculated from the other digits - allowing input validation. The numerical version of the barcode is often printed below in case the barcode cannot be read. Each digit in the barcode is unique, both in the correct orientation and inverted 180\textdegree. Barcode readers generally indicate that the scanning has been successful either visually or using a beep.
\subsubsection{Types Of Barcode Reader}
Barcode readers are very common, the technology within them is reliable and well understood. Laser scanners are the most common type of barcode reader. Consumer devices will usually have multiple lasers, improving usability.
\subsubsection{QR Codes}
\textit{Quick Response} codes are typically used to provide a link to a particular page on a website or to provide a link to an email or to some information that the app holds (for example a ticket). QR codes can store up to 7089 numbers, due to this high capacity, some businesses are replacing barcodes with QR codes. QR code scanners do not use lasers, rather QR codes can be captured with any normal camera then the app interprets the areas of light and dark as 0s and 1s. 
\subsection{Selection Of An Input Device}
The main criteria for selecting appropriate methods for data input are: cost; speed; accuracy; and reliability. Any method which involves human will probably perform poorly because humans cannot maintain the same standard of work as they get distracted, tired or are lazy. Therefore, whenever humans have to be involved in data entry, it is recommended that data is validated - this is commonly done by double entry where users have to enter their information once then enter it again and the input is rejected if the two inputs are different. 
\subsection{Other Types Of Input Device}
\subsubsection{Biometric Techniques}
Biometrics are physical characteristics (for example fingerprint; face; and voice). Biometric authentication is becoming more common especially as biometric scanners are becoming more readily available. Biometric spoofing is a term used to describe a method of fooling biometric scanners. Any biometric data collected must be done so lawfully and with the consent of the person whose data it is.
\subsubsection{Sensors}
Sensors are found commonly in control systems which act as analogue inputs. The analogue signal is converted to a digital signal using an ADC (Analogue To Digital Converter).
\subsubsection{Digital Cameras}
These are used to capture images. Most mobile phones have a digital camera built in. When the capture button is pressed, the shutter opens, this allows light to enter the camera through the lens (which focuses light onto an image sensor, made up from millions of photosites, each representing one pixel). The photosites measure the intensity of the light, a Bayer filter is used to determine the colour of each pixel. There are twice as many green as blue and red filters, as this matches the colour sensitivity of the human eye. A demosaicing algorithm is used to determine the colours of the pixels; it takes into account the intensity of light and of each of the colours on the target pixel and its neighbours. The finished image will usually be compressed, however this can be altered on different devices.
\subsubsection{RFID}
\textit{Radio Frequency Identification} allows data to be transmitted wirelessly over radio waves. There are two components to an RFID system - tag and reader. RFID tags are made up of an antenna and a chip; each tag will have a unique identifier and will often store some additional data in a non-volatile memory cell. The RFID reader transmits an encoded radio signal to interrogate the tag, which receives the signal then responds with its identifier and any other data stored (for example, stock number of product ID). Readers can process multiple tags at the same time as each has a unique identifier. There are two types of tag - active and passive. Active tags have a small battery within them whereas passive systems use the energy from the radio signal to activate the chip. 

\section{Output Devices}
There are many different output devices: speakers; printers; plotters; 3D printers; and display screens.
\subsection{Specialised Output Devices}
Some people are not able to use standard output devices due to a physical disability. Refreshable Braille displays can be used to output information to someone who has impaired vision. Although not specialised, speakers or other sound output devices can be extremely important to people with visual impairments, allowing them to use screen readers or text-to-speech utilities. 
\subsection{Printers}
There are a number of factors which should be taken into consideration when selecting a printer: speed; print resolution (measured in dpi - dots per inch); quality of colour reproduction; paper handling; and the cost of consumables. Impact printers (make multiple copies of the same document using carbonated paper) were commonplace many years ago, however now it is cheaper to make multiple copies by selecting more than one copy or photocopying the original using a modern printer. Nowadays, there are two common types of printer - inkjet and laser.
\subsubsection{Inket Printer}
These use liquid ink to produce black-and-white or colour prints. Liquid ink produces richer colours, making inkjets more suitable for printing photo-quality images. This type is usually the choice of printer in the home because they have a low up front cost; however the printer cartridges are extremely expensive - sometimes cited as the most expensive commodity in the world.
\subsubsection{Laser Printer}
These use toner (powdered ink) to produce black-and-white or colour images. These are commonly found in the workplace. Laser printers are generally good at producing text but are not so good at producing images as it is hard to reproduce deep colours with powdered ink. Whilst the powdered ink is cheap, there are other consumables which will need to be replaced from time to time. 
\subsection{Displays}
Most digital devices have some form of display. There are a number of factors to consider when choosing a display: resolution; colour reproduction; and size. One type of display is an LCD, these are rapidly being superseded by LED displays which are more environmentally friendly to run than LCDs. A variation on traditional ED displays are OLED screens, which do not use a backlight and don't require glass on the front, therefore making them lighter and more flexible. 
\subsection{Motors And Actuators}
Analogue outputs are usually controlled by motors or actuators. Digital signals are converted to analogue signals using an ADC (Analogue To Digital Converter).
\subsection{Selection Of An Output Device}
Most output devices have specialised criteria which must be taken into account when selecting an appropriate output device. Overall, cost and compatibility will generally need to be considered before purchasing. 

\section{Secondary Storage}
Secondary Storage is needed to store data which is not currently use within the CPU. It needs to have a high capacity, and will generally need to be low cost so that lots can be used in a computer. Secondary Storage may be installed internally or may be an external device. Most computers will have at least one type of secondary storage. 
\subsection{Factors To Consider When Selecting Secondary Storage}
When selecting a secondary storage device, it is important to consider the devices: capacity; speed; cost; portability; and compatibility. Another factor which \textit{can be }less important is longevity and reliability of the device. Some devices were commonplace a few years ago and now they are less common (for example, optical disk drives in computers), this means that form of media will be harder to use with newer devices. Devices can fail at any point in time, therefore it is important to have backups of data. 
\subsection{Types Of Secondary Storage}
\subsubsection{Magnetic Disk}
These disks represent binary information using two magnetic states: polarised and unpolarised. When the read-write head passes over a polarised region, it can be read as 1 and when it passes over an unpolarised region, it can be read as 0; thus allowing data to be stored in binary form. The most common type of magnetic storage is a hard disk drive. 
\subsubsection{Optical Disk}
There are three different forms of optical disks, each have a different combination of being able to be overwritten or not. They all fundamentally work by using a high-powered laser to `burn' data onto the disk, which is done by making them less reflective. There is a single binary track which spirals around the disk. Optical storage is very cheap to produce and easy to send through the post. However, it can be corrupted or damaged easily by sunlight or scratches.
\subsubsection{Flash Memory}
This is fast and compact. THe technology makes uses of NAND and NOR gates (made up from silicon semiconductors) to store electrical charge in order to store one of two values, high or low. Information is stored in blocks which are combined to form pages. Flash memory can be erased and reprogrammed electronically, and is non-volatile. Flash memory is more compact than optical or magnetic memory. It is also harder to corrupt the data stored on it. A common type of flash memory is a solid state disk, these are a good replacement for hard disk drives however they are more expensive per gigabyte than HDDs.

\section{RAM and ROM}
Computers have two kinds of internal memory - RAM and ROM.
\subsection{RAM}
\textit{Random Access Memory} is used to store programs and data that are currently bring used. It is volatile.
\subsection{ROM}
\textit{Read Only Memory} is used to store information that permanently needs to be in memory. For example, the BIOS in a computer. In embedded systems, the software never changes or rarely changes therefore it can be stored in ROM.

\section{Virtual Storage}
The use of virtual storage is becoming more common, this is where rather than using a local storage drive (for example, a hard drive in their computer), they use a cloud based storage facility (for example, Google Drive). Advantages of using virtual storage include: reduced hardware costs; improved reliability and performance; changes in scale over time. However, to access files stored in the cloud, you will need a fast internet connection. This can be quite costly. 





\end{document}


t(a) How different input, output and storage devices can be applied to the solution of different problems.
t(b) The uses of magnetic, flash and optical storage devices.
t(c) RAM and ROM.
(d) Virtual storage.