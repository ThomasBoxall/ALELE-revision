\documentclass[a4paper, 11pt, twocolumn]{article}
\usepackage[margin=0.8in]{geometry}
\usepackage{xcolor}
\usepackage{graphicx} %package to manage images
\graphicspath{ {./images/} }

\title{Further Mains Power Supplies\footnote{Images from Wjec E-book}}
\author{Revision sheet}
\date{}


\begin{document}
    
    \maketitle

    \section{Recap from AS}
    \includegraphics[width=0.4\textwidth]{recapFromAS.jpg} \\
    There are a few issues with this design, specifically the regualtion subsystem:
    \begin{itemize}
        \item The Zener diode can't sustain output current \& voltage when a drop in line voltage occurs
        \item Lots of power has to be dissipated in the resistor and zener diode.
    \end{itemize}

    \section{Improvements on regulation}
    The performance of a Zener voltage regulator can be improved by imcorporating an emitter follower into the output circuit. This reduces the power rating of the zener diode and resistor.
    \includegraphics[width=0.4\textwidth]{emitterFollower1.jpg} \\
    $V_{out} = V_Z - 0.7$\\
    $I_C = I_L$\\
    This design can be improved further by using an Op-amp.

    \section{Op-amp stabilised power supply}
    \includegraphics[width=0.4\textwidth]{opampStabilised.jpg} \\
    $V_1 = V_{out} \times \frac{R_1}{R_f + R_1}$ \\
    $V_{out} = (1+\frac{R_f}{R1})$ \\
    Advantages of using this design:
    \begin{itemize}
        \item Op-amp keeps voltage very stavble
        \item Op-amp suppkies base current
        \item Op-amp draws no current
    \end{itemize}

    \section{Load and Line regulation}
    \textbf{Line regulation }measures the ability of the power supply to maintain a steady output voltage when the input line voltage changes. \\
    \textbf{Load regulation }measures the ability of the output voltage to remain constant when the output current changes due to a change in the load.





\end{document}