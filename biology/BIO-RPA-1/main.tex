\documentclass{thomasClass}

\title{\textbf{Required Practical 1}
\\The effect of a substrate on an enzyme controlled reaction}
\author{Thomas Boxall}
\date{October 2020}

\begin{document}
\maketitle

\section{Objectives of the experiment}
 Investigate the effect of a named variable \textit{(concentration of Hydrogen Peroxide)} on the rate of an enzyme controlled reaction.
 
\section{Information about the experiment}
\begin{itemize}
    \item The substrate is Hydrogen Peroxide ($H_2O_2$)
    \item The enzyme is catalyse. This was extracted from the juice of a potato
    \item Oxygen is given off as a gas (as a bi-product of the reaction), this causes the catalyse soaked paper disc to float up and out of the solution in the test tube.
\end{itemize}

\section{Procedure}
\subsection{Method from the Practical Handbook}
\begin{enumerate}
    \item Using a marker pen write an ‘X’ on the glass halfway down one side of each of three test tubes.
    \item Add $10 cm^3$ of the solution of milk powder to each of these three test tubes.
    \item Add $2 cm^3$ of trypsin solution to $2 cm^3$ of pH 7 buffer in another set of three test tubes.
    \item Stand the three test tubes containing the solution of milk powder and the three test tubes containing trypsin and buffer in a water bath at 20 °C.
    \item Leave all six tubes in the water bath for 10 minutes.
    \item Add the trypsin and buffer solution from one test tube to the solution of milk powder in another test tube and mix thoroughly.
    \item Put the test tube back into the water bath.
    \item Repeat steps 6 and 7 using the other test tubes you set up.
    \item Time how long it takes for the milk to go clear.  Do this by measuring the time taken to first see the ‘X’ through the solution.
    \item Record the time for each of the three experiments.
    \item Using the same method, find out how long it takes the trypsin to digest the protein in the solution of milk powder at 30 °C, 40 °C, 50 °C, 60 °C.
\end{enumerate}

\subsection{Procedure notes}
\begin{itemize}
    \item To change the concentration of the hydrogen peroxide, add else $H_2O_2$ and add more water to the test tube.
    \item After fully submerging the disc, we shake it because it helps to make it more even.
    \item The monitored variable (temperature) was identified as this because we need to know if it fluctuates lots throughout the experiment. To measure this, we could use a thermometer.
    \item To keep the temperatures constant, we could put the test tube in a water bath at a constant temperature.
    \item When repeating readings, we need to make sure that the results are close to each other.
\end{itemize}

\section{Data}
\subsection{Concentration values of Hydrogen Peroxide}
\begin{table}[H]
    \centering
    \begin{tabularx}{0.55\textwidth}{c|c|c}
        \% of $H_2O_2$ & Vol. of $H_2O_2$ (ml) & Vol. of water (ml) \\
        \hline
        100 & 10 & 0 \\
        80 & 8 & 2 \\
        60 & 6 & 4 \\
        40 & 4 & 6 \\
        20 & 2 & 8
    \end{tabularx}
    \caption{Concentration values of Hydrogen Peroxide}
    \label{tab:concentrationCalculations}
\end{table}
\subsection{Results from experiment}
\textit{NB: This experiment was repeated twice.}
\begin{table}[H]
    \centering
    \begin{tabularx}{0.8\textwidth}{X|X|X|X|X}
        Conc. of $H_2O_2$ (\%) & Time taken (exp. 1) (s) & Time taken (exp. 2) (s) & Mean (s) & Rate of $O_2$ production \\
        \hline
        100 & 11 & 6 & 8.5 & 0.1176 \\
        80 & 8 & 11 & 9.5 & 0.1053 \\
        60 & 18 & 11 & 14.5 & 0.06897 \\
        40 & 13 & 28 & 20.5 & 0.0488 \\
        20 & 24 & 50 & 37 & 0.0270
    \end{tabularx}
    \caption{Results from experiment}
    \label{tab:results}
\end{table}

\subsection{Graph}
\begin{figure} [H]
    \centering
    \includegraphics[width=0.7\textwidth]{RPA-1-GRPAH.pdf}
    \caption{Graph (Scanned from Practical book)}
\end{figure}


\section{Evaluation}
\begin{itemize}
    \item Some anomalies were generated as I didn't start the timer at the right time.
    \item Concentrations of $H_2O_2$ solution may not be 100\% accurate because the measuring cylinder is quite hard to pour into. It would have been better to use a pipette.
    \item In an ideal world, it would have been better to repeat the experiment multiple times. Because of the nature of this experiment - you should make up new $H_2O_2$ solutions each time.
\end{itemize}
At low concentrations, the concentration is a limiting factor for the rate of reaction because there isn't enough substrate, to collide with enzymes, to allow the reaction to happen quickly. At a high concentration; there are lots of substrate molecules so E-S complexes can be formed more easily. This will be limited by the number of enzymes.

\end{document}
