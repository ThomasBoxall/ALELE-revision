\documentclass{thomasClass}
\usepackage[utf8]{inputenc}

\title{\textbf{Required Practical 7}
\\Use of chromatography to investigate the pigments isolated from leaves of different plants e.g. leaves from shade-tolerant and shade intolerant plants or leaves of different colours
}
\author{Thomas Boxall}
\date{September 2021}

\begin{document}

\maketitle

\section{Method}
\begin{enumerate}
    \item Set-up two boiling tubes. Add $3cm^3$ of solvent to both of them. Label one A and the other B
    \item Take a bit of leaf and grind it up with a pestle and mortar. Add some extraction solvent if needed
    \item Take a piece of chromatography paper that fits into the boiling tube and fold the top over then stick a pin through it, into the underside of the bung. At the other end of the paper, rule a line across, about 2cm up. This is the origin line
    \item Using the spotting pipette, add a few drops of the leaf onto the origin line. Make sure to dry the spot in between drops
    \item Gently lower the chromatography paper into the boiling tube, making sure the solvent in the tube doesn't doesn't cross the origin line
    \item Put the boiling tube in a rack and leave for 20 minutes
    \item Repeat steps 2-6 for the other type of leaf.
\end{enumerate}

\section{Risk assessment}
\begin{table}[H]
\centering
\begin{tabularx}{0.8\textwidth}{XXXX}
Substance / source & Hazard & Risk & Control \\
\hline
\hline
Leaves & Plant material; irritant; toxic & May cause irritation & Ensure proper identification of plants; wash hands after handling; observe normal hygiene precautions \\
\hline
Glassware & Sharp if broken & Possibility of cuts & Correct disposal if broken \\
\hline
Chromatography Solvent & Highly flammable; harmful; serious health hazard; hazardous to environment & Harmful if swallowed; irritating to eyes/skin; risk of serious damage to eyes and lungs; vapours may cause drowsiness/dizziness & Run chromatograms in stoppered/covered containers; use fume extractors/cupboards and ventilation \\
\hline
Drawing Pins & Sharp instruments & Possibility of piercing skin & Handle with care \\
\hline
Cork Borer & Sharp instrument & Possibility of injury & Handle with care, aim away from hands
\end{tabularx}
\end{table}

\section{Results}
\begin{table}[H]
    \centering
    \begin{tabularx}{0.8\textwidth}{X|XX}
         & Spinach & Money Plant \\
        \hline
        Distance moved by solvent front & 7.2cm & 9.3cm \\
        Distance moved by spot & 7.2cm & 9.3cm \\
        Rf Value & 0.8 & 0.99
    \end{tabularx}
\end{table}
I calculated the Rf value using the following equation:
$Rf = \frac{\mbox{Distance moved by spot}}{\mbox{Distance moved by solvent front}} $

\section{Evaluation}
I think that the experiment went quite well, however I think that next time we need to add more leaf juice onto the paper to increase the amount which can move, increasing the likely hood of seeing different colours. Another improvement would be to use less extraction solvent, making the leaf juice thicker.

\section{Reference}
AQA Biology - A level Year 2, 2nd Edition. Toole and Toole, Oxford

\end{document}
