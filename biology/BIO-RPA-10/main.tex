\documentclass{thomasClass}
\usepackage[utf8]{inputenc}
\usepackage{amssymb}

\title{\textbf{Required Practical 10}
\\Investigation into the effect of an environmental variable on the movement of an animal using either a choice chamber or a maze
}
\author{Thomas Boxall}
\date{November 2021}

\begin{document}

\maketitle

\section{Method}
\begin{enumerate}
    \item Setup the choice chambers with a small amount of cotton wool in the bottom of each quadrant
    \item Lay the gauze on top of the wool and put the lid on
    \item Complete Experiment 1 as follows (this is exploring light intensity)
    \begin{enumerate}
        \item Pour the 12 woodlice into the central hole and slide a food over 2 of the quadrants
        \item Leave the woodlice for 5 minutes then record how many are in each of the following halves
        \begin{itemize}
            \item Dark
            \item Light
        \end{itemize}
    \end{enumerate}
    \item Complete Experiment 2 as follows (this is exploring humidity)
    \begin{enumerate}
        \item Moisten 2 adjacent quadrants
        \item Pour the 12 woodlice into the central hole
        \item Leave the woodlice for 5 minutes then record how many are in each of the following quadrants
        \begin{itemize}
            \item Damp
            \item Dry
        \end{itemize}
    \end{enumerate}
    \item Complete Experiment 3 as follows (this is exploring humidity and light intensity together)
    \begin{enumerate}
        \item Moisten 2 adjacent quadrants (these may already be damp if you've previously conducted experiment 2)
        \item Pour the 12 woodlice into the central hole and slide the hood over one of the damp and one of the dry quadrants, creating four different environments
        \item Leave the woodlice for 5 minutes then record how many are in each of the following quadrants
        \begin{itemize}
            \item Damp and dark
            \item Damp and light
            \item Dry and dark
            \item Dry and light
        \end{itemize}
    \end{enumerate}
    \item Evaluate the results using an appropriate statistical test.
\end{enumerate}

\section{Safe handling of living organisms}
\begin{itemize}
    \item Don't eat or drink around them
    \item Cover open wounds with a plaster
    \item Wash hands thoroughly before and after the experiment
    \item Handle the organisms using appropriate implements
    \item If safe to do so, return the organism to where they came from after the experiment; if unsafe, dispose of them.
\end{itemize}
\section{Results}
\subsection{Experiment 1}
\subsubsection{Introduction}
This experiment is looking into whether woodlice prefer light or dark conditions. \\
\textbf{Null Hypothesis: }There is no preference between light and dark environments for woodlice
\subsubsection{Data}
\begin{table}[H]
\begin{tabularx}{0.6\textwidth}{X|XXXXX}
 & $O$ & $E$ & $O-E$ & $O-E^2$ & $\frac{(O-E)^2}{E}$ \\
 \hline
Light & 10 & 6 & 4 & 16 & 2.67 \\
Dark & 2 & 6 & -4 & 16 & 2.67
\end{tabularx}
\end{table}
Chi Value = 5.34
\subsubsection{Statistical Proof}
Degrees of Freedom = 1\\
$\therefore$ Critical value (at P=0.05) = 2.84 \\
As the chi value is greater than the critical value, the null hypothesis is rejected.\\

\subsection{Experiment 2}
\subsubsection{Introduction}
This experiment is investigating whether woodlice prefer wet or dry conditions. \\
\textbf{Null Hypotheses: }There is no preference between wet and dry conditions for woodlice.\\
\textit{NB: This experiment was only carried out in light conditions. For a fair test, it should have also been carried out in dark conditions.}

\subsubsection{Data}
\begin{table}[H]
\begin{tabularx}{0.6\textwidth}{X|XXXXX}
 & $O$ & $E$ & $O-E$ & $O-E^2$ & $\frac{(O-E)^2}{E}$ \\
 \hline
Wet & 12 & 6 & 6 & 36 & 6 \\
Dry & 0 & 6 & -6 & 36 & 6
\end{tabularx}
\end{table}
Chi Value = 12
\subsubsection{Statistical Proof}
Degrees of freedom = 1\\
$\therefore$ Critical value (at P=0.005) = 3.84 \\
As the Chi value is greater than the critical value, the null hypothesis is rejected.\\


\subsection{Experiment 3}
\subsubsection{Introduction}
This experiment is investigating whether woodlice prefer light and dry or light and wet or dark and dry or dark and wet conditions.\\
\textbf{Null Hypothesis: }The woodlice have no preference between the conditions listed above.
\subsubsection{Data}
\begin{table}[H]
\begin{tabularx}{0.6\textwidth}{l|XXXXX}
 & $O$ & $E$ & $O-E$ & $O-E^2$ & $\frac{(O-E)^2}{E}$ \\
 \hline
Wet/Light & 0 & 3 & -3 & 9 & 3 \\
Wet/Dark & 12 & 3 & 9 & 81 & 27 \\
Dry/Light & 0 & 3 & -3 & 9 & 3 \\
Dry/Dark & 0 & 3 & -3 & 9 & 3
\end{tabularx}
\end{table}
Chi Value = 36
\subsubsection{Statistical Proof}
Degrees of freedom = 3\\
$\therefore$ Critical value (at P=0.005) = 7.82 \\
As the Chi value is greater than the critical value, the null hypothesis is rejected.

\end{document}
