\documentclass{thomasClass}
\usepackage[utf8]{inputenc}
\usepackage{multirow}
\usepackage[fleqn]{mathtools}

\title{\textbf{Required Practical 8}
\\Investigation into the effect of a named factor on the rate of dehydrogenase activity in extracts of chloroplasts
}
\author{Thomas Boxall}
\date{September 2021}

\begin{document}
\maketitle

\section{Hypothesis}
I think that the test with ammonium hydroxide will inhibit the LIR.
\subsection{Null Hypothesis}
The named factor (temperature) will not have an effect on the rate of reaction.

\section{Variables}
\textbf{Independent Variable: }Temperature (Ice and warm water)\\
\textbf{Dependant Variable: }Rate of reaction (change in DCPIP colour intensity)\\
\textbf{Control Variables: }Volume of solutions, light intensity, temperature for the repeats

\section{Method}
\begin{enumerate}
    \item Half fill a large beaker with ice and place a small beaker on top of the ice
    \item Prepare the chloroplast medium and put it in the beaker on ice. Then remove and stand to one side
    \item Label five test tubes \textit{A, B, C, X, Y}. Stand these 5 tubes in the ice in the large beaker. Position the lamp about 10cm from the beaker so that all tubes are illuminated. Turn on the lamp.
    \item Set up the test tubes as follows
    \begin{enumerate}
        \item[(A)] Put $5cm^3$ of DCPIP; $1cm^3$ water and $1cm^3$ chloroplast suspension in the tube. Immediately wrap the tube completely in aluminium foil to exclude light
        \item[(B)] Put $5cm^3$  DCPIP solution and $1cm^3$ water and $1cm^3$ isolation medium in the tube
        \item[(C)] Put $6cm^3$ water and $1cm^3$ chloroplast suspension in the tube
        \item[(X)] Put $5cm^3$ DCPIP solution and $1cm^3$ water in the tube
        \item[(Y)] Put $5cm^3$ DCPIP solution and $1cm^3$ ammonium hydroxide in the tube
    \end{enumerate}
    \item Put tube C in the colorimeter and use it to set the absorbance to zero
    \item Add $1cm^3$ of chloroplast suspension to tube X, quickly mix the contents and start the timer. After exactly two minutes, measure the absorbance of the mixture in the colorimeter
    \item Record the data in the table
    \item Repeat steps 3 through 8 using hot water in place of ice
\end{enumerate}

\section{Data}
\begin{table}[H]
\centering
\begin{tabularx}{0.8\textwidth}{c|XXX|XXX}
\multirow{3}{*}{Tube} & \multicolumn{6}{c}{Absorbance} \\
 & \multicolumn{3}{c}{Ice (3\textdegree C)} & \multicolumn{3}{c}{Hot water (38\textdegree C)} \\
 & Data set 1 & Data set 2 & Mean & Data set 1 & Data set 2 & Mean \\
 \hline
A & 0.82 & 0.63 & 0.725 & 0.69 & 0.92 & 0.805 \\
B & 0.31 & 0.36 & 0.335 & 0.48 & 0.46 & 0.56 \\
C & 0.00 & 0.00 & 0.00 & 0.00 & 0.00 & 0.00 \\
X & 0.48 & 0.51 & 0.495 & 0.67 & 0.77 & 0.72 \\
Y & 0.55 & 0.67 & 0.61 & 0.65 & 0.85 & 0.75
\end{tabularx}
\end{table}

\section{Statistical analysis}
This data would be suitable to analyse using the T-Test
\subsection{Data}
\begin{table}[H]
\centering
\begin{tabularx}{0.8\textwidth}{X|XX|XX}
\multirow{2}{*}{Sample number} & \multicolumn{2}{c}{Test 1} & \multicolumn{2}{c}{Test 2} \\
 & Abs in cold water (4\textdegree C) & $(X-\overline{X})^2$ & Abs in warm water(38\textdegree C) & $(X-\overline{X})^2$ \\
 \hline
1 & 0.875 & 0.012 & 0.321 & 0.005 \\
2 & 0.732 & 0.001 & 0.162 & 0.008 \\
3 & 0.678 & 0.008 & 0.249 & 0.000 \\
4 & 0.811 & 0.002 & 0.310 & 0.004 \\
5 & 0.736 & 0.001 & 0.210 & 0.002
\end{tabularx}
\end{table}

\begin{figure} [H]
    \centering
    \begin{minipage}{0.45\textwidth}
        \textsc{Values for test 1}\\
        \textbf{Total: }3.832\\
        \textbf{Mean: }0.766\\
        \textbf{Standard Deviation: }0.069\\
    \end{minipage}\hfill
    \begin{minipage} {0.45\textwidth}
        \textsc{Values for test 2}\\
        \textbf{Total: }1.252\\
        \textbf{Mean: }0.250\\
        \textbf{Standard Deviation: }0.060\\
    \end{minipage}
\end{figure}

\subsection{Statistical Proof}
Now we have the values required for the T-Test, I can use the formula \\

\begin{align*}
    t&=\frac{\overline{X_1} - \overline{X_2}}{\sqrt{\frac{{S_1}^2}{n_1}+\frac{{S_2}^2}{n_2}}} \\
    t&=\frac{0.776-0.250}{\sqrt{\frac{0.069^2}{5}+\frac{0.060^2}{5}}} \\
    t&=12.618
\end{align*}
The critical value for 8 degrees of freedom at P = 0.05 is 2.31. As the t value is greater than the critical value, we reject the null hypotheses. We can therefore conclude that there is a significant difference between the two means, so temperature doesn't affect the rate of reaction, this is not by chance.

\end{document}
